\documentclass{pset}

\renewcommand{\hmwkTitle}{2nd\ week\ hw}
\renewcommand{\hmwkDueDate}{February 12, 2014}
\renewcommand{\hmwkClass}{Folland Reading Group}
\renewcommand{\hmwkClassTime}{Chapter 1}
\renewcommand{\hmwkAuthorName}{}

%
% Title Page
%

\title{
    \vspace{2in}
    \textmd{\textbf{\hmwkClass:\ \hmwkTitle}}\\
    \normalsize\vspace{0.1in}\small{Due\ on\ \hmwkDueDate\ at 3:10pm}\\
    \includegraphics[scale=0.2]{frog} \\
    \vspace{0.1in}\large{\textit{\hmwkClassTime}}
    \vspace{3in}
}

\author{\hmwkAuthorName}
\date{}

\renewcommand{\part}[1]{\textbf{\large Part \Alph{partCounter}}\stepcounter{partCounter}\\}

% Integral dx  
\newcommand{\dx}{\mathrm{d}x}

\begin{document}

\maketitle

\pagebreak 

\begin{problem}
    \begin{enumerate}[label=\alph*.]
        \item let $\mu^*(E)=\inf\bigg\{\sum_{i=1}^\i\mu_{|\mcl{A}}(A_i)\Big| E\subseteq \bigcup A_i\bigg\}$ be the outer measure induced by $\mcl{A}$ and the premeasure $\mu_{|\mcl{A}}$. Then due to proposition 1.14, $\mu^*_{|\mcl{M}}=\mu$. We could then find for each $M\in\mcl{M}$ and $\eps>0$ a sequence $\{A_i\}_{i=1}^\i\subset\mcl{A}$ such that $A=\bigcup_{i=1}^\i A_i \supset M$ and
        \[
            \mu_{|\mcl{A}}(A) < \mu^*(M)+\eps = \mu(M)+\eps
        \]
        and since $M$ is a subset of $A$, $\mu(M \symdiff A) = \mu(A)-\mu(M) < \eps$ and by the finiteness of $\mu$, the series $\sum_{i=1}^\i \mu(A_i)$ is convergent which implies there exists an $N\in\bN$ such that
        \[
            \sum_{i=N+1}^\i A_i<\eps
        \]
        which means
        \[
            \mu\Bigg(A \symdiff \bigcup_{i=1}^N A_i\Bigg) = \mu(A)-\mu\Bigg(\bigcup_{i=1}^N A_i\Bigg) \leq \sum_{i=N+1}^\i \mu(A_i)< \eps
        \]
        and now we have
        \[
            \mu\Bigg(M \symdiff \bigcup_{i=1}^N A_i\Bigg) \leq \mu(M \symdiff A) + \mu\Bigg(A \symdiff \bigcup_{i=1}^N A_i\Bigg) < 2\eps
        \]
        moreover, $\bigcup_{i=1}^N A_i \in \mcl{A}$ which gives us what we want.

        \begin{corollary}
            if $(X, \mcl{M}, \mu)$ is a $\sigma$-finite measure space, $\mcl{A}$ is an algebra that generates $\mcl{M}$ and $E$ is a subset with finite measure then for each measurable subset $F$ of $E$ and $\eps>0$ there exists a set $A \in \mcl{A}$ such that $\mu(F \symdiff A)<\eps$
        \end{corollary}
        \begin{trivialproof}
            consider $(E, \mcl{M}_E, \mu_E)$, where $\mcl{M}_E$ is the restriction of $\mcl{M}$ to $E$ and $\mu_E=\mu_{M_E}$, which is a finite measure space and $\mcl{P}(E)\cap\mcl{A}$ obviously generates $\mcl{M}_E$.

            enuf said :))))
        \end{trivialproof}
        \item consider the algebra $\mcl{A} \subset \mcl{P}(\bR)$ of bounded and co-bounded sets generated by bounded open sets. Since every open set in $\bR$ is countable union of bounded open sets: 
        \[
            O=O \cap \bR=O \cap \Big(\bigcup_{i=0}^\i (-i-1, -i+1) \cup (i-1, i+1)\Big)= \bigcup_{i=0}^\i \Big(O\cap \big((-i-1, -i+1) \cup (i-1, i+1)\big)\Big)
        \]
        that means $\mcl{A}$ generates $\mcl{B}_\bR$ but any member of $\mcl{A}$ obviously can't approximate, say, $[0, \i)$ so we're going to restrict ourselves to sets with finite measure.

        Now suppose $(X, \mcl{M}, \mu)$ is a $\sigma$-finite measure space, $\mcl{A}$ is an algebra that generates $\mcl{M}$ and $E$ is a subset with finite measure and fix $\eps>0$. There must exist a disjoint sequence $\{E_i\}_{i=1}^\i$ where $\bigcup E_i=X$ and $\mu(E_i)$ is finite for all $i$. That means
        \[
            \mu(E)=\mu\Bigl(E \cap \bigcup_{i=1}^\i E_i\Bigr)=\sum_{i=1}^\i\mu(E \cap E_i) 
        \]
        and since $\mu(E)$ is finite there must exist some $N\in\bN$ such that
        \[
            \sum_{i=N+1}^\i\mu(E \cap E_i)<\eps
        \]
        and for each $E_i$ where $1\leq i \leq N$ we can find an $A_i \in \mcl{A}$ such that $\mu\bigl(A_i \symdiff (E \cap E_i)\bigr) < 2^{-i}\eps$
        \begin{align*}
            \mu\biggl(\bigcup_{i=0}^N A_i \symdiff E\biggr) &= \mu\biggl(\bigcup_{i=0}^N A_i \symdiff \bigcup_{i=0}^\i (E \cap E_i) \biggr) \\
            &\leq \mu\biggl(\bigcup_{i=0}^N A_i \symdiff \bigcup_{i=0}^N (E \cap E_i) \biggr) + \eps \\
            &\leq \sum_{i=0}^N \mu(A_i \symdiff (E \cap E_i)) + \eps \\
            &\leq 2\eps
        \end{align*}
        \item Suppose $E \in \mcl{L}$ and $m(E)<\i$ and let $\eps>0$, because of outer regularity we can find an open set $O \supset E$ such that $m(O)<m(E)+\eps$. from that it follows
        \begin{align*}
            m(O \symdiff E) &= m(O \cup E) - m(O \cap E) \\
            &= m(O) - m(E) \\
            &< \eps
        \end{align*}
        \item suppose $\mu(E)<\i$ and $\mu(E \cap I) \leq \alpha \mu(I)$ for all intervals $I$. Put $\eps=\frac{(1-\alpha)\mu(E)}{\alpha}$. There exists a sequence of a open intervals $\{I_i\}_{i=0}^\i$ such that $E\subseteq \bigcup I_i$ and
        \[
            \mu\Bigl(\bigcup I_i\Bigr)<\mu(E)+\eps
        \]
        but we have
        \begin{align*}
            \mu(E) &= \mu\Bigl(\bigcup (E\cap I_i)\Bigr) \\
            &\leq \sum_{i=0}^\i \mu(E \cap I_i) \\
            &\leq \alpha \sum_{i=0}^\i \mu(I_i) \\
            &< \alpha\Big(\mu(E)+\eps\Big)
        \end{align*}
        which is a contradiction. Now, if $\mu(E)=\i$, since $\bR$ is $\sigma$-finite, it must have a measurable subset $E'$ such that $0<\mu(E')<\i$ for which for all $\alpha<1$ we can find an interval $I$ such that $\mu(E'\cap I)>\alpha\cdot\mu(I)$ but $\mu(E\cap I)\geq \mu(E'\cap I)$ so we're done.
    \end{enumerate}
\end{problem}
\begin{problem}
    \begin{enumerate}[label=\alph*.]
        \item follows immediately from caratheodry (see 1.11, Folland).
        \item first we note the following lemma
        \begin{lemma}
            $|\mcl{B}_\bR|=|\bR|$
        \end{lemma}
        \begin{trivialproof}
            Gomez said it's true so it must be.
        \end{trivialproof}
        and since $|\mcl{B}_\bR|=|\bR|\prec|\mcl{P}(\bR)|=|\mcl{P}(C)|$ (where $C$ is the cantor set, which is uncountable), there must exist subsets of $C$ that aren't in $\mcl{B}_\bR$ and since $\mu(C)=0$ that means $\mu$ isn't complete.
        \item It should be obvious that the completion of $(X, \mcl{B}_\bR, \mu)$ is the smallest complete space containing it. So it suffices to prove $\mcl{L} \subseteq \mcl{B}_\bR$.
        \begin{lemma}
            lebesgue measurable sets with 0 measure are subsets of borel sets with 0 measure.
        \end{lemma}
        \begin{trivialproof}
            Let $E\in\mcl{L}$ and $m(E)=0$. Because of outer regularity, we can find an open set $O_i \supset E$ such that $m(O_i)<m(E)+2^{-i}=2^{-i}$ for all $i\in\bN$. Which means $E \subseteq \bigcap_{i=0}^\i O_i$ and $m\biggl(\bigcap_{i=0}^\i O_i\biggr)=0$. And since $\bigcap_{i=0}^\i O_i \in \mcl{B}_\bR$, we're done.
        \end{trivialproof}
        now any set $L\in\mcl{L}$ can be decomposed as $L=H \cup N$ where $H \in F_\sigma \subset \mcl{B}_\bR$ and $m(N)=0$ (see 1.19, Folland) and, because of lemma 2.2, we're done.
    \end{enumerate}
\end{problem}
\begin{problem}
    \begin{enumerate}[label=\alph*.]
        \item first we construct a $\mu$ for $2$ spaces and then use induction.
        \begin{lemma}
            the set $\mcl{E}=\{E_1\times E_2 | E_1\in\mcl{M}_1, E_2\in\mcl{M}_2\}$ is an elementary family
        \end{lemma}
        \begin{trivialproof}
            it's obvious that $\varnothing \in \mcl{E}$ and if $E,F\in\mcl{E}$ then $E \cap F \in \mcl{E}$ and $(E_1 \times E_2)^c = (E_1^c \times X_2) \cup (E_1 \times E_2^c)$
        \end{trivialproof}
        and let $\mcl{A}$ be the now define the premeasure $\mu_0\bigl((E_1 \times E_2)\bigr) = \mu_1(E_1)\cdot \mu_2(E_2)$ using that we define the outer measure 
        \[
            \mu^*(E)=\inf\left\{\sum_1^\i \mu_0(E_i)\Big|E_i\in\mcl{E} \text{ and } E\subseteq\bigcup E_i\right\}
        \]
        and the restriction of that to $\mcl{M}$ is a measure.
    \end{enumerate}
\end{problem}

\end{document}
