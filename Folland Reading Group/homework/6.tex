\documentclass{pset}

\renewcommand{\hmwkTitle}{6th\ hw}
\renewcommand{\hmwkDueDate}{February 12, 2014}
\renewcommand{\hmwkClass}{Measure Theory}
\renewcommand{\hmwkClassTime}{Chapter 7}
\renewcommand{\hmwkAuthorName}{}

%
% Title Page
%

\title{
    \vspace{2in}
    \textmd{\textbf{\hmwkClass:\ \hmwkTitle}}\\
    \normalsize\vspace{0.1in}\small{Due\ on\ \hmwkDueDate\ at 3:10pm}\\
    \includegraphics[scale=0.2]{frog} \\
    \vspace{0.1in}\large{\textit{\hmwkClassTime}}
    \vspace{3in}
}

\author{\hmwkAuthorName}
\date{}

\renewcommand{\part}[1]{\textbf{\large Part \Alph{partCounter}}\stepcounter{partCounter}\\}
\newcommand{\ev}{\operatorname{ev}}

\begin{document}

\maketitle

\pagebreak
\begin{problem}
    \begin{enumerate}
        \item It's just the topology generated by $f^{-1}(O)$ for all $f\in\mcl{F}$ and $O$ is open in $Y_f$
        \item ($\Rightarrow$) fix some $x\in\mfk{X}$ and let $O$ be a neighbourhood around $L(x)$ which means that $ev_x^{-1}(O)$ is a neighbourhood around $L$ hence $L_\lambda$ is eventually in $\ev_x^{-1}(O)$ which means $L_\lambda(x)$ is eventually in $O$ and since $x$ and $O$ were arbitrary $L_\lambda$ converges to $L$ pointwise.
        
        ($\Leftarrow$) let $O$ be a neighbourhood around $L$, it must then contain an open set of the form $\ev_x^{-1}(O)$ for some $x\in\mfk{X}$ and $O$ open in $\bC$ and contains $L(x)$
    \end{enumerate}
\end{problem}
\begin{problem}
    \begin{enumerate}
        \item First, note that $0\in A$ iff  $\tau^0A=A\in M$. Now suppose $0\in A\iff\tau^nA\in M$ for all $n\leq N$ but then
        \[N+1\in A\iff N\in \tau A \iff \tau^N(\tau A)\in M \iff \tau^{N+1}A\in M\]
        hence the statement $\tau^nA\in M\iff n\in A$ is true for all $n\in\bN$

        And since $\tau^nA\in M$ iff $A\in\tau^{-n}(M)$ hence $A\in \tau^{-n}(M)\iff n\in A$ from which
        \[a,a+n,\dots,a+(k-1)n\in A \iff \tau^aA\in M\cap\tau^{-n}(M)\cap\cdots\tau^{-(k-1)n}(M)\]
        follows immediately hence $A$ contains an arithmetic progression of size $k$ iff there exists an $n\in\bN$ such that $M\cap\tau^{-n}(M)\cap\cdots\tau^{-(k-1)n}(M)$ is non-empty.
    \end{enumerate}
\end{problem}
\end{document}