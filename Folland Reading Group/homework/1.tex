\documentclass{article}

\usepackage{preamble}
\graphicspath{ {../}}
\newcommand{\hmwkTitle}{6th\ hw}
\newcommand{\hmwkDueDate}{February 12, 2014}
\newcommand{\hmwkClass}{Measure Theory}
\newcommand{\hmwkClassTime}{Chapter 7}
\newcommand{\hmwkAuthorName}{}

%
% Title Page
%

\title{
    \vspace{2in}
    \textmd{\textbf{\hmwkClass:\ \hmwkTitle}}\\
    \normalsize\vspace{0.1in}\small{Due\ on\ \hmwkDueDate\ at 3:10pm}\\
    \includegraphics[scale=0.2]{frog} \\
    \vspace{0.1in}\large{\textit{\hmwkClassTime}}
    \vspace{3in}
}

\author{\hmwkAuthorName}
\date{}

\renewcommand{\part}[1]{\textbf{\large Part \Alph{partCounter}}\stepcounter{partCounter}\\}
\newcommand{\ev}{\operatorname{ev}}

\begin{document}

\maketitle

\pagebreak 

\begin{homeworkProblem}[7]
    Let $\mu=\sum_1^na_i\mu_i$. It suffices to show $\mu$ satisfies the definition of a measure
    \begin{enumerate}
        \item \(
        \mu(\varnothing)=\sum_1^na_i\cdot 0=0
        \)
        \item let $\{E_i\}_1^\infty\subset\mathcal{M}$ be a collection of disjoint sets. It's evident that $\mu(E_j)\geq 0$ for all $j$
        \begin{align}
            \mu\lp\bigcup_1^\infty E_j\rp &= \sum_{i=1}^n\sum_{j=1}a_i\mu_i(E_j) \nonumber \\
            &=\sum_{j=1}^\i\sum_{i=1}^na_i\mu_i(E_j) \\
            &=\sum_{j=1}^\i\mu(E_j) \nonumber
        \end{align}
        (1) was just because all the terms were positive
    \end{enumerate}
\end{homeworkProblem}
\begin{homeworkProblem}
    For all $i\geq n$ for all $N\in \bN$ we have
    $$
    \bigcap_{i=1}^n\bigcup_{j=i}^\infty E_j\subseteq E_i
    $$
    which means
    \begin{align*}
        \mu\lp\bigcap_{i=1}^n\bigcup_{j=i}^\infty E_j\rp&\leq \mu(E_i) \\
        \mu\lp\bigcap_{i=1}^n\bigcup_{j=i}^\infty E_j\rp&\leq \inf\{\mu(E_i)\}_{i=N}^\infty \\
        \lim_{n\to\infty}\mu\lp\bigcap_{i=1}^n\bigcup_{j=i}^\infty E_j\rp&\leq\lim_{n\to\infty}\inf\{\mu(E_i)\}_{i=n}^\infty \\
        \mu(\liminf E_i)&\leq \liminf\mu(E_i)
    \end{align*}
\end{homeworkProblem}
\begin{homeworkProblem}
    $$
    \mu(E)+\mu(F)=\mu(E\setminus F)+\mu(F)+\mu(E\cap F)=\mu(E\cup F)+\mu(E\cap F)
    $$
\end{homeworkProblem}
\begin{homeworkProblem}
    It suffices to confirm that $\mu_A$ satisfies the definition of a measure:
    \begin{enumerate}
        \item $\mu_E(\varnothing)=\mu(E\cap\varnothing)=\mu(\varnothing)=0$
        \item let $\{E_i\}_1^\infty\subseteq\mathcal{M}$ be a collection of disjoint sets
        \begin{align*}
            \mu_E\lp\bigcup_{i=1}^\infty E_i\rp &= \mu\lp E\cap\bigcup_{i=1}^\infty E_i\rp \\
            &= \mu\lp\bigcup_{i=1}^\infty E\cap E_i\rp \\
            &= \sum_{i=1}^\infty\mu(E\cap E_i) \\
            &= \sum_{i=1}^\infty\mu_E(E_i)
        \end{align*}
    \end{enumerate}
\end{homeworkProblem}
\begin{homeworkProblem}
    it suffices to prove the ($\Leftarrow$) direction for both questions. Let $\{E_i\}_1^\infty\subseteq \mathcal{M}$ be a collection of disjoint sets
    \begin{enumerate}
        \item Suppose $\mu$ is continuous from below: take $F_n=\bigcup_{i=1}^nE_i$ that means $\{F_i\}_1^\infty$ is an increasing sequences of sets. We then have
        \begin{align*}
            \mu\lp\bigcup_{i=1}^\infty E_i\rp &= \mu\lp\bigcup_{i=1}^\infty F_i\rp \\
            &= \lim_{n\to \infty}\mu(F_n) \\
            &=\lim_{n\to\infty}\mu\lp\bigcup_{i=1}^nE_i\rp \\
            &=\lim_{n\to\infty}\sum_{i=1}^n\mu(E_i) \\
            &=\sum_{i=1}^\infty\mu(E_i)
        \end{align*}
        \item Suppose $\mu$ is continuous from above: take $F_n=\bigcap_{i=1}^nE^c_i$ that means $\{F_i\}_1^\infty$ is a decreasing sequences of sets. We then have
        \begin{align*}
            \mu\lp\bigcup_{i=1}^\infty E_i\rp &= \mu(X)-\mu\lp\bigcap_{i=1}^\infty E_i\rp \\
            &=\mu(X)-\mu\lp\bigcap_{i=1}^\infty F_i\rp \\
            &=\mu(X)-\lim_{n\to\infty}\mu(F_i) \\
            &= \lim_{n\to \infty}\mu(X)-\mu\lp\bigcap_{i=1}^nE^c_i\rp \\
        \end{align*}
        \begin{align*}
            &=\lim_{n\to\infty}\mu\lp\bigcup_{i=1}^nE_i\rp \\
            &=\lim_{n\to\infty}\sum_{i=1}^n\mu(E_i) \\
            &=\sum_{i=1}^\infty\mu(E_i)
        \end{align*}
    \end{enumerate}
\end{homeworkProblem}
\begin{homeworkProblem}
    \begin{enumerate}[label=\alph*.]
        \item suppose without loss of generality that $\mu(E)>\mu(F)$. we then have
        $$
        \mu(E\cup F)\geq\mu(E)
        $$
        and
        $$
        \mu(E\cap F)\leq \mu(F)
        $$
        but we have
        \begin{align*}
            \mu(E\symdiff F) &= \mu(E\cup F)-\mu(E\cap F) \\
            &\geq \mu(E)-\mu(F) \\
            &>0
        \end{align*}
        \item we know that $E\symdiff G\subseteq E\symdiff F\cup F\symdiff G$ so we have
        \[
        0 \leq \mu(E \symdiff G) \leq \mu(E \symdiff F \cup F \symdiff G)\leq \mu(E\symdiff F) + \mu(F\symdiff G)\leq 0 
        \]
        which means \(\sim\) is transitive and it's trivially symmetric and reflexive which makes it an equivalence relation.
        \item same as before, we know that \(E \symdiff G \subseteq E \symdiff F \cup F \symdiff G\) so that means
        \[
        \rho(E, G) = \mu(E \symdiff F) 
        \leq \mu(E \symdiff F) + \mu(F \symdiff G)
        = \rho(E, F) + \rho(F, G)
        \]
    \end{enumerate}
\end{homeworkProblem}
\end{document}
