\documentclass{article}

\usepackage[english]{babel}
\usepackage[a4paper,top=2cm,bottom=2cm,left=3cm,right=3cm,marginparwidth=1.75cm]{geometry}

% Useful packages
\usepackage{amsmath}
\usepackage{graphicx}
\usepackage[colorlinks=true, allcolors=blue]{hyperref}

\title{Measure Theory Reading Group}
\author{DarQ}

\begin{document}
\maketitle

\section{Why join this group}
I always thought the lack of the social element was the single biggest disadvantage to us self learners. It is my hope to create a social environment for people (me included) interested in learning the same topics to give them an opportunity to give and receive help and discuss and teach each other the stuff they learned and to also provide some light accountability and extra motivation and to hopefully provide a studying environment that's more fun and engaging which are all aspects conducive to learning which are absent if you're studying alone.
non-self-learners are welcome of course, but they might find it less valuable

\section{The group's format}
We're gonna closely follow Folland's "Real Analysis: Modern Techniques and Their Applications" for measure theory and occasionally use Lee's "Introduction to Topological Manifolds" as a pointset reference. We'll hold a weekly meeting where someone organizes a talk to present the week's content or solve an interesting or insightful problem or talk about some other related topic that ends in a discussion.

Other group activities will include:
\begin{itemize}
    \item problem hit list where we all agree to solve a bunch of problems together
    \item help channels where we can help each other or despair together
    \item read-along streams
\end{itemize}
We'll also have JohnDS\#1149 who will be supervising the group and providing psets, feedback, etc. :\^)
\section{Prerequisites}
real analysis at the introductory level is needed. Something like rudin 1-7 would suffice.
\iffalse
\subsection{syllabus}
This is a rough progression map showing what we're gonna hopefully cover each week.
\subsubsection{Measure theory}
\begin{itemize}
    \item Week 1 Sigma-algebras , measures and their properties. 
    \item Week 2  The outer measure (Caratheordry) , the lebesgue measure and the cantor set.
    \item Week 3  Measurable functions , Basic integration  ( MCT , Fatou's lemma) 
    \item Week 4  Complex integration , LDCT , Reimann vs lebesgue integral. 
    \item Week 5 Modes of convergence , egoroff's theorem , Product measure.
    \item Week 6 n-dimensional lebesgue integral and polar coordinates. 
    \item Week 7 signed measures ( Hahn decomposition) , Radon-Nykodym 
    \item Week 8 Complex measures and differentiation ( Lebesgue differentiation theorem) 
    \item Week 9 Functions of bounded variations 
    \item Week 10 Basic $L^p$ spaces (Holders , minkowski ) , $L^p$ spaces are banach , density results. 
    \item Week 11  Dual of $L^p$ , Chebychevs inequality , Generalized minkowsky , Distribution Functions. 
    \item Week 12 Positive Linear functions on Cc(X) and the Reisz representation theorem.

\end{itemize}
\subsubsection{Pointset}
\begin{itemize}
    \item week 1 topological spaces, convergence and continuity
    \item week 2 Hausdorff spaces, bases and countability and defining manifolds
    \item week 3 subspaces, product spaces, disjoint union spaces and quotient spaces
    \item week 4 Adjunction spaces, topological groups and group actions
    \item week 5 finish up last week stuff + problems
    \item week 6 connectedness and compactness
    \item week 7 local compactness, paracompactness
    \item week 8 proper maps + problems
\end{itemize}
\fi
\section{Sweet, where do I sign up?}
First, I ask that you only consider joining if you find yourself interested in the objective of the study group and you're able to dedicate at least a couple of hours a week to the group. If you're unsure you are you'll always be welcome when you're sure but I want an accurate headcount of the group members to make sure everyone is engaged. So as to confirm your commitment I'm gonna ask you to read chapter 1 section 1 "introduction" of Folland and then fill in \href{https://docs.google.com/forms/d/e/1FAIpQLSfBfSXGfLJCOsI43vFtWFRaY8jWbum7rY_5jI1e2Ou2RtrFgg/viewform?usp=sf_link}{this} google form and after I confirm your commitment I'll personally send you a link to the discord server where we're organizing the group :)

\section{Questions}
Feel free to DM me (DarQ\#7126) if you have any questions :\^)

\end{document}