\documentclass{pset}
\usepackage{indentfirst}
\usepackage{quiver}
\begin{document}
\section{2.1 Simplicial and Singular Homology}
\subsection{Exact Sequences and Excision}
\indent The idea of the first part of the section is to figure out a way to relate $\H A$, $\H X$ and $\H {X, A}$. The way it does that is by linking them all in a long exact sequence. It then uses \emph{excision} to assert that in some cases
\[\H {X, A} = \H {X/A}\]

The idea behind the proof of 2.21 stand on constructing geometrically motivated homomorphisms: Taking the barycentric subdeivision process which is a purely geometric construction and turning that into an algebraic construction. Basically, given a space $X$ and bunch of sets $\mcl{U}$ whose interior form an open cover of $X$ and a singular simplex $\sigma\colon\Delta^n\to X$ you can chop that simplex up continuously into smaller and smaller singular simplices till the image of each singular simplex is contained entirely in some $U$ in $\mcl{U}$. Essentially, you construct a homomorphism $\rho\colon C_n(X)\to C_n^\mcl{U}(X)$. You then prove that $\rho$ is chain homotopy equivalent to the identity, hence the morphism it induces between $\H X$ and $H_n^\mcl{U}(X)$ is an isomorphism

\subsection{Proof of The Five Lemma}
\[\begin{tikzcd}
	A & B & C & D & E \\
	{A'} & {B'} & {C'} & {D'} & {E'}
	\arrow["i", from=1-1, to=1-2]
	\arrow["j", from=1-2, to=1-3]
	\arrow["k", from=1-3, to=1-4]
	\arrow["\ell", from=1-4, to=1-5]
	\arrow["\epsilon", from=1-5, to=2-5]
	\arrow["\delta", from=1-4, to=2-4]
	\arrow["\gamma", from=1-3, to=2-3]
	\arrow["\beta", from=1-2, to=2-2]
	\arrow["\alpha"', from=1-1, to=2-1]
	\arrow["{i'}"', from=2-1, to=2-2]
	\arrow["{j'}"', from=2-2, to=2-3]
	\arrow["{k'}"', from=2-3, to=2-4]
	\arrow["\ell'"', from=2-4, to=2-5]
\end{tikzcd}\]
\begin{theorem}
    If the above diagram is commutative, the rows are exact and $\alpha$, $\beta$, $\delta$ and $\epsilon$ are isomorphisms then so is $\gamma$.
\end{theorem}
\begin{proof}
    First we proof that $\gamma$ is surjective: let $c'\in C'$, since $\delta$ is surjective there exists an element $d\in D$ such that $k'(c') = \delta(d)$. Since the diagram is commutative and the rows are exact we see that
    \[\epsilon(\ell(d)) = \ell'(\delta(d)) = \ell'(k'(c')) = 0\]
    but since $\epsilon$ is injective, $d\in\ker\ell$ hence $d\in\img k$, therefore there exists an element $c\in C$ such that
    \[\delta(k(c)) = k'(\gamma(c)) = k'(c')\]
    hence $c'-\gamma(c)\in\ker k'$ hence $c'-\gamma(c)\in\img j'$ hence there exists an element $b'\in B$ such that $j'(b') = c'-\gamma(c)$ but since $\beta$ is surjective, there exists an element $b\in B$ such that $j'(\beta(b)) = c'-\gamma(c)$ but we know that $j'(\beta(b)) = \gamma(j(b))$ hence
    \[\gamma(c-j(b)) = \gamma(c) - \gamma(j(b)) = \gamma(c)+c'-\gamma(c) = c'\]
    hence $c'\in\img \gamma$.

    The proof that $\gamma$ is also injective is prolly similar and I'm too lazy to right it out.
\end{proof}
\subsubsection{remarks}
\begin{enumerate}[label=\arabic*.]
    \item What I'm seeing from the proof of the barycentric subdeivision lemma is that proving facts about the 
\end{enumerate}
\subsubsection{questions}
\begin{enumerate}[label=\arabic*.]
    \item Are constant chains cycles?
    \begin{quote}
        The answer is yes but that's not all, they're also \emph{boundries} hence null elements in any homology group.
    \end{quote}
\end{enumerate}
\section{2.2 Computations and Applications}
\subsection{Cellular Homology}

\section{TODO}
\begin{enumerate}
	\item READ THE HOMOTOPY EXTENSION PROPERTY!!!
\end{enumerate}
\end{document}